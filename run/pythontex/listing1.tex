\documentclass{article}

\usepackage[T1]{fontenc}
\usepackage[utf8]{inputenc}
\usepackage{amsmath,amssymb}
\usepackage[russian,english]{babel}
\usepackage{listings}
\usepackage[rerun=always]{pythontex}
\usepackage[a4paper, total={6in, 8in}]{geometry}

\newcommand{\pymultiply}[2]{\py{#1*#2}}

\begin{document}
\selectlanguage{russian}

\par
Pycode
\par
\begin{pycode}
print("Python says ``Hello!''")
\end{pycode}

\par
Now trying pyv
\par

\begin{pyblock}
print("Python says Hello!")
\end{pyblock}

\par
Now pyconsole
\par
\begin{pyconsole}
print("Python says Hello!")
raise ValueError('aaaaa')
\end{pyconsole}

The document is partly in Russian, например, тут:

$8 \times 256 = \pymultiply{8}{256}$


\newcounter{codecounter}
\begin{lstlisting}[escapeinside={(*}{*)}, numbers=left]
/* Hello World program */

#include<stdio.h>
main()
{
    printf("Hello World"); (*\label{code:printf}*)
    double a = b (*\textcolor{red}{\footnote{This is a variable $b$}}*) + c;
    double d = b (*\textcolor{red}{(\refstepcounter{codecounter}\thecodecounter\label{code:b})}*) + c;
}

\end{lstlisting}

\end{document}

