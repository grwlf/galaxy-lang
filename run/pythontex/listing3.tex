\documentclass{article}

\usepackage[T1]{fontenc}
\usepackage[utf8]{inputenc}
\usepackage{amsmath,amssymb}
\usepackage[russian,english]{babel}
\usepackage{minted}
\renewcommand{\MintedPygmentize}{/nix/store/9r8n35slzprv3q4hfdnc0fz3r70z2y64-python3.7-Pygments-2.7.2/bin/pygmentize}
\usepackage[rerun=always, pygopt={texcomments=true}]{pythontex}
%\newcommand{\pymultiply}[2]{\py{#1*#2}}
% \usepackage[landscape]{geometry}
\usepackage{lscape}
\usepackage[a3paper]{geometry}
\usepackage{pdflscape}
\usepackage{longtable}

% \newenvironment{pyconcodeblock}%
%  {\VerbatimEnvironment
%   \begin{VerbatimOut}{_temp.py}}%
%  {\end{VerbatimOut}%
%   \pyconc{exec(compile(open('_temp.py', 'rb').read(), '_temp.py', 'exec'))}%
%   \inputpygments{python}{_temp.py}}


\begin{document}

% \newcolumntype{C}{>{\centering\arraybackslash}m{20mm}}

\selectlanguage{russian}

\section{Pylightnix for Nix users}

\subsection{Expressions}

\par

\begin{tabular}{|p{0.5\textwidth}|p{0.5\textwidth}|}

\hline

\multicolumn{1}{|c|}{\textbf{Nix}} &
\multicolumn{1}{|c|}{\textbf{Pylightnix}}

\\
\hline

\begin{minted}[linenos, escapeinside=!!, breaklines, breakafter=0123456789-]{nix}
{ pkgs? import <nixpkgs> {}
, stdenv ? pkgs.stdenv
}:

let
  version = "2.10";
  url = "mirror://gnu/hello/hello-\${version}.tar.gz";
  sha256 = "0ssi1wpaf7plaswqqjwigppsg5fyh99vdlb9kzl7c9lng89ndq1i";
in
with pkgs;
stdenv.mkDerivation rec {
  name = "hello";
  src = fetchurl { inherit url sha256; };

  buildInputs = [ gnutar ];

  buildCommand = ''
    tar -xzf $src
    cd hello-2.10
    ./configure --prefix=$out
    make
    make install
  '';
}
\end{minted}

&

\small
\setpythontexfv{breaklines, breakafter=0123456789-, commandchars=\\\{\}}
\begin{pyblock}[stdout][numbers=left]
from tempfile import TemporaryDirectory
from shutil import copytree
from os import getcwd, chdir, system
from os.path import join
from pylightnix import (
  Manager, DRef, RRef, Config, RefPath,
  mkconfig, Path, Build, build_outpath, mkdrv,
  build_wrapper, match_latest, dirrw, mklens,
  instantiate, realize, fetchurl, promise)

version = '2.10'
url = f'http://ftp.gnu.org/gnu/hello/hello-{version}.tar.gz'
sha256 = '31e066137a962676e89f69d1b65382de95a7ef7d914b8cb956f41ea72e0f516b'

def stage_hello(m:Manager)->DRef:

  def _config()->Config:
    name:str = 'hello-bin'
    src:RefPath = fetchurl(m, name='hello-src',
                              url=url,
                              sha256=sha256)
    out_bin = [promise, 'usr', 'bin', 'hello']
    # TODO: Filter variables of Manager type internally
    return {k:v for k,v in locals().items() if k!='m'}

  def _make(b:Build)->None:
    o:Path = build_outpath(b)
    with TemporaryDirectory() as tmp:
      copytree(join(mklens(b).src.syspath, f'hello-{version}'), join(tmp,'src'))
      dirrw(Path(join(tmp,'src')))
      cwd = getcwd()
      try:
        chdir(join(tmp,'src'))
        system('./configure --prefix=/usr')
        system('make')
        system(f'make install DESTDIR={o}')
      finally:
        chdir(cwd)

  return mkdrv(m, mkconfig(_config()),
                  match_latest(),
                  build_wrapper(_make))
\end{pyblock}

\normalsize

\\

% Nix comments

% Here goes reference to a line \ref{myline}.

&

% Pylightnix comments

\\

\hline
\end{tabular}

\subsection{Evaluation}

The following snippet illustrates how to build the Pylightnix stage defined
above.

% print(rref)
% print(mklens(rref).out_bin.syspath)
\begin{pyblock}[stdout]
from subprocess import check_output   # FU \label{fetchurl}
rref:RRef = realize(instantiate(stage_hello))
print(check_output(mklens(rref).out_bin.syspath, shell=True))
\end{pyblock}
Result:
\stdoutpythontex

Ref to \ref{fetchurl}.

\end{document}

