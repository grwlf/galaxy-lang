\documentclass{article}

\usepackage[T1]{fontenc}
\usepackage[utf8]{inputenc}
\usepackage{amsmath,amssymb}
\usepackage[russian,english]{babel}
\usepackage{minted}
\renewcommand{\MintedPygmentize}{/nix/store/9r8n35slzprv3q4hfdnc0fz3r70z2y64-python3.7-Pygments-2.7.2/bin/pygmentize}
\usepackage{pythontex}
\newcommand{\pymultiply}[2]{\py{#1*#2}}
\usepackage[a4paper, total={6in, 8in}]{geometry}

\begin{document}
\selectlanguage{russian}

\begin{pycode}
print("Python says ``Hello!''")
\end{pycode}

The document is partly in Russian, например, тут:

$8 \times 256 = \pymultiply{8}{256}$

\par
Some Python code:

\par

\begin{minted}{python}
import numpy as np

def run(genl1,genl2):
    m = len(genl1)
    n = len(genl2)
    M = None # to become the incidence matrix
\end{minted}

\par

Some NIX code

\par

\begin{minted}{nix}
{ pkgs ?  import ./3rdparty/nixpkgs {}
, stdenv ? pkgs.stdenv
} :
let
  self = pkgs.python37Packages;
in
  self
\end{minted}

\par

Now, a table

\par

\begin{tabular}{|p{0.5\textwidth}|p{0.5\textwidth}|}
% \begin{tabular}{|c|c|}
\hline

\begin{minted}{python}
import numpy as np

def incmatrix(genl1,genl2):
    m = len(genl1)
    n = len(genl2)
    M = None
    VT = np.zeros((n*m,1), int)

    #compute the bitwise xor matrix
    M1 = bitxormatrix(genl1)
    M2 = np.triu(bitxormatrix(genl2),1)
    return M
\end{minted}

&

\begin{minted}{nix}
{ pkgs ?  import ./3rdparty/nixpkgs {}
, stdenv ? pkgs.stdenv
} :
let
  self = pkgs.python37Packages;
in
  self
\end{minted}


\\

C

&

D

\\

\hline
\end{tabular}

\end{document}

