\documentclass{article}

\usepackage[T1]{fontenc}
\usepackage[utf8]{inputenc}
\usepackage{amsmath,amssymb}
\usepackage[russian,english]{babel}
\usepackage{listings}
\usepackage{pythontex}
\usepackage{makecell}
\newcommand{\pymultiply}[2]{\py{#1*#2}}

\begin{document}
\selectlanguage{russian}

\begin{pycode}
print("Python says ``Hello!''")
\end{pycode}

The document is partly in Russian, например, тут:

$8 \times 256 = \pymultiply{8}{256}$



\begin{tabular}{|c|c|}
\hline

\newcounter{codecounter}
\begin{lstlisting}[escapeinside={(*}{*)}, numbers=left]
/* Hello World program */

#include<stdio.h>
main()
{
    printf("Hello Left World"); (*\label{code:printf1}*)
    double a = b (*\textcolor{red}{\footnote{This is a variable $b$}}*) + c;
    double d = b (*\textcolor{red}{(\refstepcounter{codecounter}\thecodecounter\label{code:b})}*) + c;
}
\end{lstlisting}

&

\newcounter{codecounter2}
\begin{lstlisting}[escapeinside={(*}{*)}, numbers=left]
/* Hello World program */

#include<stdio.h>
main()
{
    int foo=1;
    int bar=33;
    printf("Hello Right World"); (*\label{code:printf2}*)
    double a = b (*\textcolor{red}{\footnote{This is a variable $b$}}*) + c;
    double d = b (*\textcolor{red}{(\refstepcounter{codecounter2}\thecodecounter\label{code:b})}*) + c;
}
\end{lstlisting}
\\
We produce the output in line \ref{code:printf1}.
Fooo Baaar
&
And this output we produced in line \ref{code:printf2}.
\\

\hline
\end{tabular}

XXX \footnote{This is a variable $b$}}

\end{document}

